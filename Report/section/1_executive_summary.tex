\section{Executive summary}

Greengers è una \textit{startup} nata nel 2021 che si occupa di \textit{coltivazione sostenibile} e formata da cinque giovani ragazzi: Raffaele, Matteo, Mattia, Primiano e Luca.

Nel seguente report presentiamo il nostro progetto \textbf{PotNet}, un prodotto innovativo che supporta le persone nell'approccio alla coltivazione.
Siamo partiti da un'analisi dettagliata del problema che vogliamo risolvere con il nostro prodotto, dopo aver condotto alcune piccole indagini sulla attitudine a le abitudini di un esiguo campione di persone in merito al tema coltivazione.

Terminata questa fase di analisi, considerando i dati emersi dal sondaggio, abbiamo iniziato a pensare ad un dispositivo innovativo, installabile su vasi già esistenti, che fornisca assistenza e monitoraggio durante la crescita delle piantine.
Per questo motivo, siamo passati ad analizzare il settore di riferimento per capire la possibile redditività del nostro prodotto e i principali competitors e prodotti sostitutivi.

A questo punto, abbiamo esaminato i bisogni del mercato identificando dei possibili target di riferimento per il nostro dispositivo ed eseguendo alcune stime di mercato.
Individuate delle possibili categorie di acquirenti, ci siamo concentrati sulle modalità di vendita del prodotto al cliente finale, stimando anche dei possibili costi finali in considerazione con quelli di produzione.

Terminate le fasi di analisi di settore e di mercato, non rimaneva che delineare in modo preciso le funzionalità che PotNet, avviando, quindi, l'analisi tecnologica del prodotto. A questo proposito, ha iniziato a prendere forma una primo schizzo di architettura logico-fisica, con una prima forma di progettazione del sistema software.

Al fine di ottimizzare le successive fasi di produzione e realizzazione, ci siamo dedicati alla pianificazione delle attività necessarie partendo dalla progettazione dettagliata dei sistemi.

Mentre il sistema software iniziava a diventare una realtà, grazie alla collaborazione con \textit{Almalabor}, siamo riusciti a realizzare anche il primo prototipo fisico di PotNet, in cui è stato integrato anche la primissima build del substrato software.

