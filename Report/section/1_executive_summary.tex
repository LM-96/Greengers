\section{Executive summary}

Greengers è una \textit{startup} nata nel 2021 che si occupa di \textit{coltivazione sostenibile} e formata da cinque giovani ragazzi: Raffaele, Matteo, Mattia, Primiano e Luca. Nel seguente report presentiamo il nostro progetto \textbf{PotNet}, un prodotto innovativo che aiuta le persone a prendersi cura delle proprie piante.

Siamo partiti da un'analisi dettagliata del problema che vogliamo risolvere e, dopo aver condotto alcune piccole indagini sui bisogni delle persone nell'ambito delle piante casalinghe, siamo giunti alla conclusione che in molti avrebbero bisogno di un supporto e di maggiori conoscenze per prendersi cura al meglio delle proprie piante.

Terminata questa fase di analisi, considerando i dati emersi dalle nostre indagini, abbiamo iniziato a pensare ad un dispositivo intelligente, installabile su vasi già esistenti, che fornisca assistenza e monitoraggio durante la crescita delle piante.\\Dunque, siamo passati ad analizzare il settore di riferimento per capire la possibile redditività del nostro prodotto, i principali competitors e se fossero già presenti prodotti simili sul mercato.

A questo punto, abbiamo identificato dei possibili target di riferimento per il nostro dispositivo ed eseguito alcune stime di mercato.\\Individuate delle possibili categorie di acquirenti, ci siamo concentrati sulle modalità di vendita del prodotto, stimando anche i costi di produzione e come contenere quelli di sviluppo. 

Terminate le fasi di analisi di settore e di mercato, abbiamo delineato con precisione le funzionalità di PotNet e avviato l'analisi tecnologica del prodotto. Parallelamente, è partita la progettazione dell'architettura logico-fisica e la realizzazione del sistema software di base.

Al fine di ottimizzare le fasi di progettazione e realizzazione, abbiamo identificato le attività essenziali e le abbiamo suddivise tra i membri del team in modo da parallelizzare il più possibile le attività elementari del progetto.\\Per tenere sotto controllo la pianificazione delle attività e il loro stato di avanzamento, a partire dalla progettazione fino al lancio del prodotto sul mercato, abbiamo delineato un diagramma di Gannt. 

Una volta messo a punto un planning efficiente, grazie alla collaborazione con \textit{Almalabor}, siamo riusciti a realizzare un prototipo fisico di PotNet, in cui è stato integrato una versione alpha del substrato software.\\Il prototipo è servito, e servirà, per validare tutte le funzionalità identificate in fase di progettazione e per risolvere eventuali criticità non riscontrate in fase di design.

I prototipi ottenuti nelle ultime fasi di test saranno molto simili al prodotto finale, pertanto verranno utilizzate a scopo dimostrativo per campagne pubblicitarie. La fase di promozione e pubblicità sarà utile a far conosce il prodotto al cliente finale e ad aziende interessate a future collaborazioni.