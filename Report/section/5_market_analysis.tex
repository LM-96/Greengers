\newpage
\section{Analisi del mercato e profilazione del cliente}

\subsection{Raccolta dati}

Per raccogliere i dati sui bisogni del mercato, si è scelto di eseguire delle interviste utilizzando gli strumenti messi a disposizione da Google Forms. In particolare, si sono raccolte, su un campione totale stimato di cinquecento persone a cui è stato inviato il questionario, 269 opinioni di possibili futuri acquirenti.
Dai dati raccolti si è potuto osservare che:

\begin{itemize}
	\item Il 65\% di persone intervistate è in possesso di piante nella propria abitazione e di questi il 40\% afferma di trascurare la cura delle proprie piante a causa di dimenticanze o di scarso tempo a disposizione;
	
	\item Il 34\% degli intervistati che non possiedono piante si dicono poco consoni all’acquisto perché non saprebbero come prendersene cura;
	
	\item Il 71\% degli intervistati si è detto interessato ad un prodotto che li supportasse nella cura delle proprie piante.
\end{itemize}

\subsection{Target di riferimento}

Il nostro prodotto, PotNet, è indirizzato principalmente ad un \textbf{pubblico di età compresa tra i 18 e i 45 anni} (73\% degli intervistati) e abbiamo identificato i seguenti gruppi di acquirenti:

\begin{itemize}
	\item \textbf{Professionals}, come piccoli o grandi agricoltori;
	
	\item \textbf{Plants enthusiasts}, ossia i possessori di piante da appartamento o piccoli orti e hanno una conoscenza sufficiente a prendersi cura delle loro piante;	
	
	\item \textbf{Black thumbs}, quella categoria di persone che non hanno abbastanza conoscenze per potersi prendere cura di una pianta;
	
	\item \textbf{Busy workers}, sono tutte quelle persone che non riescono a prendersi cura di una pianta a causa del tempo o di dimenticanze dovute dal loro stile di vita.
\end{itemize}

L’utilizzo del nostro prodotto non è limitato alla sola sfera privata, ma \textbf{è estendibile per essere utilizzato nel mondo business}, andando ad includere piccole-grandi aziende agricole tra i nostri target.