\section{Definizione del problema da risolvere}

Già negli anni '80, si iniziò a definire il concetto di \textit{agricoltura di precisione}: un insieme di strategia e strumenti in grado di migliorare la produttività del suolo grazie all'impiego di sistemi satellitari e georefernziati. \`E stato, quindi, l'inizio del binomio \textit{tecnologia-agricoltura}, rafforzato con il termine \textit{argicoltura digitale}, fino ad arrivare all'\textit{agricoltura 4.0}\footnote{\url{https://www.bioaksxter.com/it/nuove-tecnologie-inizia-l-era-dell-agricoltura-4-0?module=blog&operazione=nuove-tecnologie-inizia-lera-dellagricoltura-4-0}}.

Questa innovazione tecnologica, perlomeno inizialmente, ha interessato specialmente le \textit{grandi coltivazioni}, come quelle di aziende agricole specializzate, o grandi orti, anche se negli ultimi anni si stanno affacciando sul mercato alcuni tentativi di portare queste innovazioni anche all'interno di ambienti domestici.

Moltissimi privati, nella propria abitazione, dispongono di piante o vorrebbero averle, specialmente in questi tempi in cui la pandemia e le restrizioni dovute ad essa hanno fatto si che la coltivazione di piccoli spazi verdi, abbia assunto un ruolo fondamentale per i cittadini. Infatti, secondo il sito \href{https://www.businessintelligencegroup.it/quanto-vale-il-mercato-del-giardinaggio-in-italia/}{businessintelligencegroup}, "\textit{oltre il 60\% delle persone che possiede un’area green (giardino, ma anche terrazzo, veranda o orto) la apprezza e la sfrutta con molta più frequenza rispetto al passato}"

Tuttavia, le piante sono organismi delicati che, dunque, hanno bisogno di cure costanti, di attenzioni ma anche di competenze. In quest'ottica, la diffusione di dispositivi intelligenti a supporto della \textit{coltivazione domestica} di piccole piante, è fondamentale per supportare gli appassionati, ma anche le persone che vorrebbero approcciarsi al mondo \textit{green}.

Una piccola indagine svolta da noi mostra che, su un campione di 268 privati, quasi il 35\% non ha una piantina in casa e che, di questi, la maggior parte non ha tempo o competenze.
Risulta chiaro, allora, che molte persone \textit{vorrebbero} avere una o più piantine in casa, ma non possono a causa dei motivi che abbiamo appena mostrato.

In quest'ottica, nasce un forte bisogno di qualcosa che possa aiutare queste persone a colmare le carenze di tempo e competenze, in modo da permettere anche a loro di godere dei benefici che scaturiscono dalla coltivazione di una pianta, senza, però, fargli \textit{perdere il contatto} con il senso stesso di quest'arte: in pratica, serve un supporto che \textit{semplifichi e automatizzi} la coltivazione domestica, senza, però, sostituirsi completamente all'uomo.

Se, infatti, coloro che non sono immersi nel mondo dell'agricoltura lamentano l'assenza di tempo e competenze, anche gli appassionati potrebbero trarre notevole beneficio da meccanismi di aiuto alla crescita delle proprie piantine, dato che potrebbero dargli l'opportunità di estendere \textit{il proprio orticello} e, magari, anche il bagaglio di conoscenze permettendo di sperimentare \textit{nuovi mondi verdi}.



\begin{enumerate}
	\item \textbf{Pochi prodotti \textit{entry level}}\\
	
	\item \textbf{Tempo e competenze necessarie}\\
	
	\item \textbf{Scarsa adattabilità}\\
\end{enumerate}
