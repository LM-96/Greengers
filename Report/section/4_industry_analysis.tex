\newcolumntype{l}{>{\hsize=.3\hsize}X}
\newcolumntype{d}{>{\hsize=.18\hsize}X}
\newcolumntype{s}{>{\hsize=.14\hsize}X}

\newcolumntype{L}{>{\centering\arraybackslash}l}
\newcolumntype{D}{>{\centering\arraybackslash}d}
\newcolumntype{S}{>{\centering\arraybackslash}s}

\renewcommand\tabularxcolumn[1]{m{#1}}

\newpage
\section{Analisi del settore (analisi competitiva)}

Dopo aver svolto un'attenta analisi sulle dinamiche della nostra startup e dopo aver avuto anche un riscontro da parte dell'ISTAT, si è arrivati alla conclusione che il nostro settore di riferimento è quello delle \textbf{attività di ricerca e sviluppo sperimentale nel campo delle altre scienze naturali e dell'ingegneria}, con codice ATECO \textbf{72.19.09}\footnote{\url{https://www.codiceateco.it/categoria/ricerca-e-sviluppo-sperimentale-nel-campo-delle-altre-scienze-naturali-e-dellingegneria}}. Svolgendo delle ricerche usando la banca dati AIDA si è visto come questo sia un settore che, in Italia, conta all'attivo 5081 tra grandi aziende e altre start-up di piccole e medie dimensioni. Tra queste, 432 aziende operano sul suolo emiliano e 73 si trovano a Bologna.

\subsection{Redditività del settore}

Utilizzando i dati estrapolati dalla sopracitata ricerca nel database AIDA si è riscontrato che nel 2020 il settore della ricerca e sviluppo sperimentale nel campo delle scienze naturali e dell'ingegneria ha prodotto un ricavo pari ad 1,8 miliardi di euro, registrando nel periodo che va dal 2016 ad oggi una crescita cumulata annua del 4,66\%.

Le aziende leader di questo settore sono:
\begin{itemize}
	\item \textbf{Nuovo Pignone Tecnologie – S.R.L.}\\
	Quest’azienda si occupa attualmente della realizzazione di compressori alternativi, turbine a gas e compressori centrifughi per la movimentazione di idrocarburi e gas da parte di General Electric.
	Nel 2020 ha registrato un fatturato di 350 milioni di euro;
	
	\item \textbf{Iqvia RDS Italy S.R.L.}\\
	Quest’azienda collabora con diversi business a livello mondiale e sviluppa soluzioni innovative nell’ambito sanitario.
	Nel 2020 ha registrato un fatturato di 57 milioni di euro;
	
	\item \textbf{Generali Jeniot S.P.A.}\\
	Quest’azienda ricerca e sviluppa soluzioni innovative nell’ambito dell’Internet of Things e della Connected Insurance.
	Nel 2020 ha registrato un fatturato di 54 milioni di euro;
	
	\item \textbf{Cisco Photonics Italy S.R.L.}\\
	Quest’azienda si occupa della produzione di componenti elettronici quali antenne, switch e guide d’onda.
	Nel 2020 ha registrato un fatturato di 37 milioni di euro.	
\end{itemize}

Queste aziende rappresentano il 28\% dell'intero mercato, indice che il settore delle attività di ricerca e sviluppo sperimentale nel campo delle scienze naturali e dell’ingegneria è abbastanza frammentato.

\subsection{Principali competitors}
\begin{center}
\begin{tabularx}{\textwidth}{| S | D | S | L | L |}
  \hline
  \textbf{AZIENDA} &
  \textbf{PRODOTTO} &
  \textbf{COSTO} &
  \textbf{FUNZIONALIT\`A} &
  \textbf{LIMITI} \\	
  \hline
 Xiaomi & Mi Flower Care Plant Sensor & 19,95€ &
- Monitoraggio di umidità e fertilità del terreno, temperatura e luce solare;\newline
- Dispone di un’applicazione mobile per visionare i dati raccolti;\newline
- Database con 6000+ piante;\newline
- Storico dei valori della pianta;\newline
- Integrabile con sistemi di Smart Home & 
- Non ha un sistema di notifiche integrato quando i valori superano dei limiti preimpostati;\newline
- Non è possibile accedere ai dati tramite PC\newline
- Non dispone di connettività Wi-Fi \\	\hline
 Wanfei & Flower Care Soil Tester & 32,99€ &
- Monitoraggio di umidità e fertilità del terreno, temperatura e luce solare;\newline
- Compatibile con l’app ‘Flower Care’;\newline
- Connettività al dispositivo mobile tramite Bluetooth; &
- Non ha un sistema di notifiche integrato;\newline
- Non dispone di connettività Wi-Fi;\newline
- Si può accedere ai dati solo tramite app mobile;\newline
- Prezzo più elevato rispetto la concorrenza \\	\hline
\end{tabularx}
\end{center}

\subsection{Potenziali entranti}

Facendo degli studi su altri settori affini a quello in cui la nostra startup lavora, abbiamo identificato alcuni potenziali entranti:
\begin{itemize}
	\item \textbf{Keepy S.R.L.}\\
	Keepy S.R.L. è un’impresa che sviluppa tecnologie, sia software che hardware, da integrare con le esistenti soluzioni relative alle serrature intelligenti per gestire gli accessi e permettere l’apertura di queste serrature da remoto;
	\item \textbf{SoftLabs S.R.L.}\\
	Softlabs S.R.L. è una compagnia che opera nel mondo delle tecnologie digitali e sviluppo di software business, specializzata anche in consulenza tecnologica e project management. 
\end{itemize}

\subsection{Prodotti sostitutivi}
I prodotti sostitutivi identificati sono:
\begin{itemize}
	\item \textbf{Vasi smart}\\
	I vasi smart sono una categoria di prodotto con dei costi che possono variare tra i 20€ e i 70€. I vasi smart più economici e con un prezzo comparabile a quello di PotNet sono dotati dei soli sensori di umidità del terreno e in alcuni casi di auto-irrigazione passiva. Anche i vasi smart più costosi, non offrono la stessa adattabilità e lo stesso livello di interazione che può offrire PotNet,
	\item \textbf{Serre smart da interno}\\
	Le serre smart da interno offrono un numero di funzionalità e servizi più ampio rispetto al nostro prodotto, ma sono economicamente meno accessibili e hanno un ingombro di spazio maggiore rispetto a PotNet.
\end{itemize}

\subsection{Fornitori}
Per la realizzazione del nostro prodotto sono stati identificati alcuni elementi chiave, scegliendo aziende strategiche necessarie per la produzione di alcuni componenti:
\begin{itemize}
	\item \textbf{TROLL System}\\
	TROLL System è un’azienda che si occupa della progettazione e della produzione di PCB. Qui potrà essere commissionata la realizzazione della "main board" di PotNet;
	
	\item \textbf{Tedicon}\\
	Tedicon è specializzata nella realizzazione di apparecchiature per la rilevazione della temperatura e di termoresistenze. Potrà essere il nostro fornitore principale per i sensori di temperatura di PotNet;
	
	\item \textbf{Bernabe Giorgio S.R.L.}\\
	La Bernabe Giorgio S.R.L. produce microcontrollori e diversi tipi di sensori per il controllo dei parametri ambientali. I sensori necessari a ricavare i dati dell’ambiente circostante la pianta, come luce solare e umidità, potranno essere forniti da questa attività.
	
	\item \textbf{VIPIEMME S.P.A.}\\
	Vipiemme S.P.A. è un’impresa specializzata nel produrre una vasta gamma di batterie. Sarà possibile commissionare a questa azienda la fornitura di batterie per alimentare il nostro dispositivo.
\end{itemize}

Dopo un'attenta analisi, si può affermare che la nostra startup potrà avere un forte potere contrattuale con queste aziende, visto l'alto numero di possibili fornitori di questi componenti.